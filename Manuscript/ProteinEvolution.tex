\documentclass[10pt]{article}

\usepackage{float}
\usepackage[T1]{fontenc}
\usepackage[utf8]{inputenc}
\usepackage[english]{babel}
\usepackage{amssymb,amsfonts,amsmath,amsthm}
\usepackage{graphicx}
\usepackage{lmodern}
\usepackage{mdframed}
\usepackage{hyperref}
\usepackage{cases}
\usepackage{framed}
\usepackage{pdfpages}
\usepackage{multicol}
\usepackage[margin=60pt]{geometry}
\usepackage{abstract}
\renewcommand{\abstractnamefont}{\normalfont\Large\bfseries}
\renewcommand{\abstracttextfont}{\normalfont\normalsize}

\usepackage{pgfplots}
%\usepgfplotslibrary{colormaps}
\pgfplotsset{every axis/.append style={line width=1pt}}

\usepackage{adjustbox}
\newcommand{\specialcell}[2][c]{%
	\begin{tabular}[#1]{@{}c@{}}#2\end{tabular}}

\newcommand{\avg}[1]{\left< #1 \right>} % for average
\newcommand{\Ne}{N_\mathrm{e}}

% \renewcommand{\baselinestretch}{2}

\begin{document}
	
	\section*{Causes of evolutionary rate variation among protein sites, Echave \textit{et al}, 2016, Nature}
	\begin{itemize}
		\item 	The rate of evolution varies among sites within proteins owing to structural and functional constraints.
		
		\item The main pattern of variation is due to structural constraints: evolutionary rates increase from the slowly evolving, solvent-inaccessible, tightly packed and rigid protein interior, to the rapidly evolving, solvent-exposed and loosely packed protein surface.
		
		\item Functional constraints result in the slow evolution of sites that are directly involved in protein function and their neighbours. There may also be longer range effects on distant sites.
		
		\item According to mechanistic biophysical models, site-specific evolutionary rates are related to mutational changes of thermodynamic stability. Structural predictors, such as solvent accessibility and local packing, would be proxies of mutational stability changes.
		
	\item	Our understanding of rate variation among sites remains limited: at best, current models explain approximately 60\% of the observed variance in site-specific rates, and in many cases these models explain considerably less.
		
		\item To make further progress, we need to develop better rate inference methods, complete the list of structural and functional molecular features that correlate with rates, and undertake further research on theoretical models derived from first principles.
	\end{itemize}
	
\end{document}
